% arara: lualatex
\documentclass[a4paper,10pt]{article}
\usepackage{geometry}
\newgeometry{margin=0.5cm}
\usepackage[T1]{fontenc}
\usepackage[dvipsnames]{xcolor}
\usepackage[lite,eucal]{mtpro2}
\usepackage{mathtools,amssymb,amsthm,dsfont}
\usepackage{varwidth}
\usepackage{tcolorbox}
\tcbuselibrary{theorems,skins}
\usepackage{varwidth}
\tcbuselibrary{skins}
\newtcbtheorem{YetAnotherDefinition}{Definición}%
{enhanced,frame empty,interior empty,colframe=ForestGreen!50!white,
	coltitle=ForestGreen!50!black,fonttitle=\bfseries,colbacktitle=ForestGreen!15!white,
	borderline={0.5mm}{0mm}{ForestGreen!15!white},
	borderline={0.5mm}{0mm}{ForestGreen!50!white,dashed},
	attach boxed title to top center={yshift=-2mm},
	boxed title style={boxrule=0.4pt},varwidth boxed title}{theo}

\begin{document}
\begin{YetAnotherDefinition}{Conjuntos abierto, cerrado, acotado, compacto y su interior y clausura de $C\subset\mathds{R}^{n}$}{mittelwertsatz_n3}%
	Recordemos que un subconjunto $V\subset\mathds{R}^{n}$ es
	\begin{description}
		\item[abierto] de $\mathds{R}^{n}$ si una de las siguientes propiedades se sigue:
		      \begin{itemize}
			      \item $V$ es vacío.
			      \item Para cualquier $x\in V$, existe un $r(x)=r>0$ tal que $B\left(x,r\right)\subset V$, el cual denota la \emph{bola abierta} que se define como \[ B\left(x,r\right)\coloneqq\LEFTRIGHT\{\}{y\in\mathds{R}^{n}\colon\|y-x\|<r\,} \]
			            donde $\|w\|\coloneqq\LEFTROOT{-2}\SQRT[2]{\sum_{i=1}^{n}w^{2}_{i}}\,$ para $w=\left(w_{1},\ldots,w_{n}\right)$.
		      \end{itemize}
		\item[cerrado] si su complemento $\mathds{R}^{n}\setminus V$ es un conjunto abierto de $\mathds{R}^{n}$.
		\item[acotado] si existe algún $r>0$ tal que $V\subset B\left(0,r\right)$.
		\item[compacto] si este es cerrado y acotado.
	\end{description}
	Dado un subconjunto $C\subset\mathds{R}^{n}$, su
	\begin{description}
		\item[interior,] denotado por $\operatorname{int}\left(C\right)$, es el conjunto abierto más grande contenido en $C$, es decir, \[ \operatorname{int}\left(C\right)=\bigcup_{\mathclap{\substack{C\supset U\\U\text{ abierto}}}} U \]
		\item[clausura,] denotado por $\overline{C}$, es el conjunto cerrado más pequeño que contiene a $C$, es decir, \[ \overline{C}=\bigcap_{\mathclap{\substack{C\subset F\\F\text{ cerrado}}}} F \]
	\end{description}
	\[  \xl\sum\nolimits_{i \notin I}\xl\int\limits_{0}^{\infty}\iint_{2}^{0}\cwoint \]
\end{YetAnotherDefinition}

\begin{YetAnotherDefinition}{Continuidad de una función}{}
	Una función $f\colon\mathcal{U}\rightarrow$
\end{YetAnotherDefinition}

\end{document}